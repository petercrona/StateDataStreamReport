\documentclass[a4paper,12pt]{article}
\usepackage{listings}
\usepackage{graphicx}
\usepackage{color}
\usepackage{afterpage}
\usepackage{amsmath}

\definecolor{lightgray}{rgb}{.9,.9,.9}
\definecolor{darkgray}{rgb}{.4,.4,.4}
\definecolor{purple}{rgb}{0.65, 0.12, 0.82}

\begin{document}

\graphicspath{ {./images/} }
\lstdefinelanguage{JavaScript}{
  keywords={typeof, new, true, false, catch, function, return, null, catch, switch, var, if, in, while, do, else, case, break},
  keywordstyle=\color{blue}\bfseries,
  ndkeywords={class, export, boolean, throw, implements, import, this},
  ndkeywordstyle=\color{darkgray}\bfseries,
  identifierstyle=\color{black},
  sensitive=false,
  comment=[l]{//},
  morecomment=[s]{/*}{*/},
  commentstyle=\color{purple}\ttfamily,
  stringstyle=\color{red}\ttfamily,
  morestring=[b]',
  morestring=[b]"
}

\lstset{
   language=JavaScript,
   extendedchars=true,
   basicstyle=\footnotesize\ttfamily,
   showstringspaces=false,
   showspaces=false,
   numbers=left,
   numberstyle=\footnotesize,
   numbersep=9pt,
   tabsize=2,
   breaklines=true,
   showtabs=false,
   captionpos=b
}

\title{Structuring asynchronous requests in AngularJS 1.3}
\date{November 16, 2014}
\author{Peter Crona\\ IceColdCode.com}
\maketitle

\begin{abstract}
This paper presents a new way of structuring asynchronous requests in AngularJS 1.3. The approach is
called \emph{StateDataStream}. Today it is common that developers mix data loading and data processing in a way that reduces readability and flexibility. 
This papers evalulates different ways of structuring asynchronous requests, starting with callbacks, then promises and finally the StateDataStream.
Both callbacks and promises have drawbacks, some of which StateDataStream solves.
\end{abstract}
\clearpage

\afterpage{\null\newpage}
\clearpage

\tableofcontents
\clearpage

\afterpage{\null\newpage}
\clearpage

\section{Background}
Two very common techniques for structuring asynchronous requests are using \emph{callbacks} and using \emph{promises}.
We will define these in short. However, let us first look at the changes in requirements as more and more transition to single page web applications.
In the past it was common to structure application as:

\begin{figure}[!htbp]
  \includegraphics[scale=0.5]{traditional_cs_model.png}
  \caption{
    Traditionally the web server has contained logic and then delivered a rendered page to the user.
    The rendered page contained most, if not all, data that the user requested.
  }
\end{figure}

\begin{figure}[!htbp]
  \includegraphics[scale=0.5]{new_cs_model.png}
  \caption{
    Today it is getting more and more common to develop single page web applications. 
    Single page web application can be considered standalone applications delivered by a web server, which has the only responsibility to deliver files.
    The web server does not contain the application logic, the logic resides in the frontend and the APIs used by the frontend.
  }
\end{figure}

A single page web application greatly increases the number of AJAX-requests required. In AngularJS a controller typically
starts with loading data from external APIs. Traditionally data was only loaded as a response to events in the frontend.
For example if a user clicked on `Show more posts`, an AJAX-request would have been sent to load more posts.
With a single page app we need to do an AJAX-request just to have any posts at all. We are not given any data initially.
This greatly increases number of AJAX-requests we need to do and thus makes it interesting to investigate new ways of structuring them.
Now let us have a look at the two main approaches used today, namely callbacks and promises.

\subsection{Callbacks}
The basic idea behind using callbacks is to run call a method with another method as argument. When the first method has finished it will call the method you provided as an argument.
It is very simple to use, but it has some drawbacks when used for more advanced data loading. Consider this trivial example where we just send one request to get some data.

\begin{lstlisting}[caption=Code showing how we can load data using callbacks, frame=single]
Api.getUserInfoById(43, 
                    userInfoResponseHandler, 
                    errorHandler);

function userInfoResponseHandler(response) { ... }
\end{lstlisting}
Now imagine that we have two \emph{dependent} pieces of data that we want. Consider the case where a part of the user's information is
his or hers Github username. After loading the user's info we want to load the user's repos on Github.

\begin{lstlisting}[caption={Code, using callbacks, showing how we can load two pieces of data, where the second is dependent on the first}, frame=single]
Api.getUserInfoById(43, 
                    userInfoResponseHandler, 
                    errorHandler);

function userInfoResponseHandler(response) {
  var githubName = response.githubName;

  // Possibly do more stuff with userInfo response here.

  Api.getGithubRepos(githubName, 
                     githubResponseHandler, 
                     errorHandler);
}

function githubResponseHandler(response) { ... }
\end{lstlisting}
This code is a quite unclear. Getting an overview of what data is loaded is difficult. And we will either have to mix data loading and data processing (do stuff in userInfoResponseHandler) or we
have to send all data to the githubResponseHandler and process the data there. Both ways result in a quite unclear structure. Why should the response handler for loading Github repos be responsible for handling user data. The structure is not a result of what we want to express, it is a result of how callbacks work. Clearly it has bad readability, but it also has bad flexibility. It is difficult to load more data, for example if we would have wanted the number of contributors for the Github repos. Furthermore, it does not make it easy to load data in parallel and then run some code when all data has been loaded. Now let us have a look at how we could accomplish the same with promises.

\subsection{Using promises}
First a brief description of what promises are. Instead for directly letting you specify a callback a so called promise is returned. The promise is an object with methods where you can register callbacks.
In AngularJS one of the most important methods of promises is `then`.
\[ then(successHandler, errorHandler) \]
In addition to just being another way of specifying callbacks the successHandler or errorHandler will always be called after the data is loaded. 
Even if you already called then or if the data was loaded before you called `then`. 
Now that you know a little about promises, consider the case where we just want to load one piece of data, the user's info.

\begin{lstlisting}[caption=Code showing how we can load the user's info with promises, frame=single]
Api.getUserInfoById(43)
   .then(userInfoResponseHandler)
   .catch(errorHandler);

function userInfoResponseHandler(response) { ... }
\end{lstlisting}
\clearpage
If we have two dependent pieces of data it might look like this:

\begin{lstlisting}[caption={Code showing an approach using promises where we are loading two pieces of data, where the second is dependent on the first}, frame=single]
Api.getUserInfoById(43)
   .then(userInfoResponseHandler)
   .catch(errorHandler);

function userInfoResponseHandler(response) {
  var githubName = response.githubName;

  // Possibly do more stuff with userInfo response here.

  Api.getGithubRepos(githubName)
     .then(githubResponseHandler), 
     .catch(errorHandler);
}

function githubResponseHandler(response) {
  ...
}
\end{lstlisting}
This is very similar to the callback method. However, there are more ways of solving the same problem with promises.
Another example where we use the possibility to chain promises is:

\begin{lstlisting}[caption={Code where we use promises and the possibility to chain promises to load two pieces of data, where the second is dependent on the first}, frame=single]
Api.getUserInfoById(43)
   .then(loadGithubRepos)
   .then(githubResponseHandler)
   .catch(errorHandler);

function loadGithubRepos(response) {
  var githubName = response.githubName;

  // Possibly do more stuff with userInfo response here.

  return Api.getGithubRepos(githubName);
}

function githubResponseHandler(response) {
  ...
}
\end{lstlisting}
Note that we have a much clearer code. By just looking at the first block of code we can see that we are getting user info, loading github repos and then handling the
response of them. But we still lack a separation of data loading and processing. One possible way of accomplishing this is:

\begin{lstlisting}[caption={Code where we chain promises together and separate data loading from data processing. However, note that we use variables in the scope outside where we load our data.}, frame=single]
var userInfo = null; var gitRepos = null;

Api.getUserInfoById(43)
   .then(loadGithubRepos)
   .then(githubResponseHandler)
   .then(processData)
   .catch(errorHandler);

function loadGithubRepos(response) {
  userInfo = response;
  return Api.getGithubRepos(response.githubName);
}

function githubResponseHandler(response) {
  gitRepos = response;
}

function processData() { Do stuff with userInfo and gitRepos }
\end{lstlisting}
But this approach have two drawbacks. Firstly, we use variables in an outside scope as storage, which can damage readability and makes it more difficult to reason about the code since the functions are less pure (rely on data not given in the parameters and have side effects).
Secondly, we treat the loading of user info differently, despite that it is just loading data just as loadGithubRepos.
Another approach without these drawbacks (I'm using Angular's \$q in this pseudo code) is:

\begin{lstlisting}[caption={Code where we are using promises and have separated data loading from data processing as well as made it very clear what data we are loading}, frame=single]
$q.when({})
  .then(loadUserInfo)
  .then(loadGithubRepos)
  .then(processData)
  .catch(errorHandler);

function loadUserInfo(state) {
  return Api.getUserInfoById(43).then(function(res) {
    state.userInfo = res;
    return state;
  });
}


function loadGithubRepos(state) {
  return Api.getGithubRepos(state.userInfo.githubName)
            .then(function(res) {
               state.githubRepos = res;
               return state;
            });
}

function processData(state) {
  console.log(state.userInfo);
  console.log(state.githubRepos);
}
\end{lstlisting}
By just looking at the chain of promises we can directly see that we are loading user info, then github repos and finally we process the data.
All data loading is treated the same. We can easily add more `loaders` if we need to load more data. However, this solution is not perfect.
It is unclear what loadUserInfo does, we need to look at the actual function to know that it writes to `userInfo` in the state. Furthermore,
it is unclear that loadGithubRepos must be after loadUserInfo. We need to look at both loadUserInfo and loadGithubRepos to know that there
is a dependency between them. Furthermore, if we would like to load independent data this would not be done in parallel. 
StateDataStream currently attacks the first and last problem: that it is unclear where in the state functions are writing (to which property) and that data is not loaded in parallel even if independent.
The second problem, with unclear dependencies between elements in the promise chain, is discussed in Conclusions and future considerations (section \ref{sec:future}).
\clearpage
\section{The StateDataStream way}
StateDataStream is solving the problem that it is unclear where the result of a promise is written when multiple promises are chained together. 
It also makes it easy to run independent AJAX-requests in parallel. Since the syntax is quite clear let us start with an example:

\begin{lstlisting}[caption={Code showing an approach using StateDataStream where we are loading two pieces of data, where the second is dependent on the first.}, frame=single]
StateDataSteam.init({})
  .write('userInfo', Api.getUserInfoById(43))
  .write('githubRepos', function(state) {
     return loadGithubRepos(state.userInfo.githubName);
  })
  .error(errorHandler)
  .execute(proceessData);

function processData(state) {
  console.log(state.userInfo);
  console.log(state.githubRepos);
}
\end{lstlisting}
We have achieved a clear separation between data loading and data processing. Furthermore, it is clear where we are writing the results of promises. `userInfo` and other promises directly added to the stream (i.e. not wrapped in a function) will be sent in paralell as soon as we add them to the stream. Hence loading independent data in parallel is easy and does not damage the readability of the code.

Now let us have a look at the details of StateDataStream, how we specify what data to load and how to handle errors. And finally also how to execute the stream, making it possible for us to use all the data we have written into the state.

\subsection{Specifying the stream}
Initially we just specify the stream. The idea is that nothing we do should have side effects. However, as you may have noted, promises directly written to the
stream will be sent immediately, since the parameters of functions are evaluated directly. Fortunately, due to how promises work, we can disregard from this. 
The following sub sections will describe the operations available for specifying the stream.

\subsubsection{Writing to the stream}
As you have seen we can write different things to the stream. A semi-formal description of the write operation is:
\[ write(key, val) \]
where
\begin{flalign}
\begin{aligned}
val &:= value\ |\ promise\ |\ function \nonumber \\
key &:= objectRef\ |\ listRef \nonumber
\end{aligned}
\end{flalign}
\emph{objectRef} and \emph{listRef} use dot notation to specify where to put the key. Both use the same syntax except that listRef always end with []. An example will suffice to describe their syntax:
\newline

\begin{lstlisting}[caption=Examples of listRef and objectRef keys, frame=single]
StateDataSteam.init({})
  .write('userInfo', val) // objectRef
  .write('user.info', val) // objectRef
  .write('user.says.hello', val) // objectRef
  .write('users[]', val) // listRef
  .write('data.users[]', val); // listRef

  // Let ? symbolise any data, then the state will look like:
  {
     userInfo: ?,
     user: {
       info: ?,
       says: {
         hello: ?
       }
     },
     users: [?],
     data: {
       users: [?]
     }
  }
\end{lstlisting}
We also need to define \emph{function} a little more. A function is just an ordinary function, but it may return a promise, in which case the promise will be resolved and written to
the stream. And if a value (anything except a promise) is returned, the value will be written to the stream.

\subsubsection{Error handling}
A stream is associated with one error handler. The error handler will immediately be called if some HTTP-request returns another status than 200. Meaning that any subsequent
operations will not be carried out. The error handler is called with the error and the state at the time of the error as arguments.
\[ errorHandler(error, state) \]

The error handler is associated with the stream as following:
\begin{lstlisting}[caption=Attaching an error handler to the stream, frame=single]
StateDataSteam.init({})
  .write('userInfo', val)
  .error(errorHandler);
\end{lstlisting}
Now let us move on to actually doing something with the stream, in contrast to just specifying it.

\subsection{Executing the stream}
Once the stream is specified it can be stored in a variable or you might create a function which returns the stream parameterised by some arguments.
Regardless of which, the way to execute the stream is simply to call execute:

\begin{lstlisting}[caption=Executing the stream, frame=single]
StateDataSteam.init({})
  .write('userInfo', val)
  .error(errorHandler)
  .execute(initController);
\end{lstlisting}
The function initController will be called with the resulting state of running the stream, assuming that an error is not detected, in which case the error handler will be called instead. 
The handler initController shall have the following signature:
\[ initController(state) \]
and in this case state will just be an object containing the property \emph{userInfo}. Note that the state is completely defined by the inital state and what we write to the state.
By just looking at the stream specification we get a good image of what will be available in initController.

\section{Conclusion and future considerations}
\label{sec:future}
StateDataStream provides a new way of structuring asynchronous requests. It separates data loading from data processing and it has a very clear syntax, which makes it easy to see 
what data is loaded and how to access it in the data processing step. Furthermore, it makes it easy to load independent data in parallel and it is easy to specify one 
error handler for the whole stream. It solves some of the drawbacks with callbacks and promises, which are commonly used today. However, it does not yet support making it easy to indentify dependencies between write operations nor loading dependent data in parallel
in an easy way.

\subsection{Specifying dependencies between writes}
\label{sec:spec_dependencies}
Consider the following piece of code:

\begin{lstlisting}[caption=Executing the stream, frame=single]
StateDataSteam.init({})
  .write('users[]', Api.getUsers())
  .write('githubRepos[]', function(state) {
     // Load github repos for all users
  })
  .error(errorHandler)
  .execute(initController);
\end{lstlisting}
It is not obvious that gitHubRepos depends on the data in state.users. We need to look at the code. If the function for loading data is not specified inline (with an anonymous function) this will significantly damage readability.

On possible way to make dependencies more clear is to introduce a third parameter for the write method. In that case we might get something like:

\begin{lstlisting}[caption=Executing the stream, frame=single]
StateDataSteam.init({})
  .write('users[]', Api.getUsers())
  .write('githubRepos[]', loadGithubRepos, ['users']);
\end{lstlisting}
In the code above it is clear that githubRepos depend on users. Furthermore, loadGithubRepos does not need to be called with the full
state, it can be called with only the specified dependencies.

\clearpage

\subsection{Running state dependent writes in parallel}
Consider the following code:
\begin{lstlisting}[caption=Executing the stream, frame=single]
StateDataSteam.init({})
  .write('users[]', Api.getUsers())
  .write('githubRepos[]', function(state) {
     // Load github repos for all users
  })
  .error(errorHandler)
  .execute(initController);
\end{lstlisting}
Currently there is no easy way of loading github repos for all users in parallel. It might be doable using \$q.all in the write. But it might be worth investing some time into investigating 
the best approach for this. One possible way to simplify this might be:

\begin{lstlisting}[caption=Executing the stream, frame=single]
StateDataSteam.init({})
  .write('users[]', Api.getUsers())
  .writeAll('githubRepos[]', loadGithubRepos, [users])
  .error(errorHandler)
  .execute(initController);
\end{lstlisting}
Where \emph{writeAll} would cause loadGithubRepos to be called once for every element in users, and the results would be pushed into githubRepos, in parallel of course.
Note that this would work with the method of specifying dependencies presented in section \ref{sec:spec_dependencies}.

\end{document}
